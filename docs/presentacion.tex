% Hecho para el FLISOL 2012
% Abril 2012
% Author: Javier Wilson
% Guegue Comunicaciones
% www.guegue.com/~javier/flisol2012
% https://github.com/javierwilson/flisol2012
%
%
\documentclass{beamer}
\usetheme{Warsaw}
\usepackage[utf8]{inputenc}
\usepackage[T1]{fontenc}
\usepackage{hyperref}

\begin{document}
\titlegraphic{\includegraphics[width=60mm]{imgs/guegueflisol.png}}
\title{Escalando sesiones en aplicaciones web}
\subtitle{o cómo compartir sesiones entre Django y PHP usando Redis}
\author{Javier Wilson <javier@guegue.net>}
\institute{Guegue Comunicaciones\\FLISOL 2012}
\date{\today}

\begin{frame}
\titlepage
\end{frame}

\begin{frame}\frametitle{Indice de contenido}\tableofcontents
\end{frame} 


\section{Intro - Galletas} 
\subsection{Sobre galletas} 
\begin{frame}\frametitle{Definiendo\ldots}
\begin{itemize}
\item Qué es una sesión?
\begin{block}{\ldots de Wikipedia}
``En telecomunicaciones, una sesión es una serie de interacciones entre los dos puntos finales de comunicación que se producen durante el lapso de una sola conexión.''  En el web, una sesión es un tipo de galleta (cookie). \pause
\end{block}
\item Qué es un cookie o galleta?
\begin{figure}
\includegraphics[height=20mm,width=60mm]{imgs/galleta1.png}
\end{figure}
\end{itemize} 
\end{frame}

\subsection{Seguridad galletas} 
\begin{frame}\frametitle{Seguridad de galletas}
Seguridad de galletas
\begin{itemize}
\item Contenido proveido por el sitio \pause
\item Sólo se envía al sitio que la creó \pause
\item Residen en la máquina del cliente (bajo su control)
\end{itemize} 
\end{frame}

\subsection{Sesiones}
\begin{frame}\frametitle{Galleta de sesión}
Qué es un session cookie?
\begin{figure}
\includegraphics[height=50mm,width=60mm]{imgs/galleta2.png}
\caption{Ejemplo de una sesión}
\end{figure}
\end{frame}

\section{Escalabilidad} 
\subsection{Más que galletas} 
\begin{frame}\frametitle{Cuando las típicas cookies no son suficiente}
\begin{itemize}
\item Qué es escalabilidad? \pause
\item PHP: usa filesystem \pause
\item Django: usa BD (django\_seession) \pause
\item Pros y Contras
\end{itemize}
\end{frame}

\subsection{Usando BD para sesiones} 
\begin{frame}\frametitle{Usando PostgreSQL para sesiones}
Vamos a usar PostgreSQL (porque nos gusta :)
\begin{itemize}
\item Modificando el manejo de sesiones \pause
\item Tabla SQL a crear \\
\begin{block}{PostgreSQL}
CREATE TABLE sessions (session\_id varchar(32) PRIMARY KEY, session\_data text, session\_expiration timestamp);
\end{block}
\pause
\item Funciones del manejador de funciones \\
\url{https://gist.github.com/1776966} \pause
\item Una opción más sencilla. php.ini: \\
session.save\_handler = redis
\end{itemize} 
\end{frame}

\subsection{Buscando alternativa a SQL (NoSQL)} 
\begin{frame}\frametitle{Buscando alternativa a SQL (NoSQL)}
\begin{itemize}
\item Simpleza de un diccionario \pause
\item Berkeley DB no habla TPC/IP (no ``sirve``) \pause
\item Redis :)
\end{itemize} 
\end{frame}

\section{Una propuesta de solución}
\subsection{Tecnologías a utilizar}
\begin{frame}\frametitle{Tecnologías a utilizar}
\begin{itemize}
\item Cómo almacenar y compartir datos de sesión: JSON \pause
\item Comenzando con PostgreSQL \pause
\item Re-utilizando y escribiendo el código \pause
\item Pasando a Redis\ldots
\end{itemize} 
\end{frame}

\subsection{Resultado}
\begin{frame}\frametitle{Resultado}
Screenshot a la mitad del camino:
\begin{figure}
\includegraphics[height=52mm,width=43mm]{imgs/dosgalletas.png}
\caption{Galletas co-habitando\ldots}
\end{figure}
\end{frame}

\begin{frame}\frametitle{Resultado}
Screenshot final: 
\begin{figure}
\includegraphics[height=30mm,width=83mm]{imgs/phpdjango.png}
\caption{Probando el código final\ldots}
\end{figure}

Código: \url{https://github.com/javierwilson/flisol2012}
\end{frame}

\end{document}
