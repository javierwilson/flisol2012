% Hecho para el FLISOL 2012
% Abril 2012
% Author: Javier Wilson
% Guegue Comunicaciones
% www.guegue.com/~javier/flisol2012
% https://github.com/javierwilson/flisol2012
%
%
\documentclass{beamer}
\usetheme{Warsaw}
\usepackage[utf8]{inputenc}
\usepackage[T1]{fontenc}

\begin{document}
\title{Escalando sesiones en aplicaciones web}
\subtitle{o cómo compartir sesiones entre Django y PHP usando Redis}
\author{Javier Wilson}
\date{\today} 

\begin{frame}
\titlepage
\end{frame}

\begin{frame}\frametitle{Indice de contenido}\tableofcontents
\end{frame} 


\section{Intro - Galletas} 
\subsection{Sobre galletas} 
\begin{frame}\frametitle{Definiendo\ldots}
\begin{itemize}
\item Qué es una sesión? \pause
\item Qué es un cookie o galleta? PIC1
\end{itemize} 
\end{frame}

\subsection{Seguridad galletas} 
\begin{frame}\frametitle{Seguridad de galletas}
Seguridad de galletas
\begin{itemize}
\item Contenido proveido por el sitio \pause
\item Sólo se envía al sitio que la creo \pause
\item Residen en la máquina del cliente (bajo su control)
\end{itemize} 
\end{frame}

\subsection{Sesiones}
\begin{frame}\frametitle{Galleta de sesión}
Que es un session cookie?
PIC2
\end{frame}

\section{Escalabilidad} 
\subsection{Más que galletas} 
\begin{frame}\frametitle{Cuando las típicas cookies no son suficiente}
\begin{itemize}
\item Que es escalabilidad? \pause
\item PHP: usa filesystem \pause
\item Django: usa BD (django\_seession) \pause
\item Pros vs Cons
\end{itemize} 
\end{frame}

\subsection{Usando BD para sesiones} 
\begin{frame}\frametitle{Usando PostgreSQL para sesiones}
Vamos a usar PostgreSQL (porque nos gusta :)
\begin{itemize}
\item Modificando el manejo de sesiones \pause
\item Tabla SQL a crear \pause
\item Funciones del manejador de funciones \pause
\item Una opcion más sencilla
\end{itemize} 
\end{frame}

\subsection{Buscando alternativa a SQL (NoSQL)} 
\begin{frame}\frametitle{Buscando alternativa a SQL (NoSQL)}
\begin{itemize}
\item Simpleza de un diccionario
\item Redis
\end{itemize} 
\end{frame}

\section{Una propuesta de solución}
\subsection{Tecnologías a utilizar}
\begin{frame}\frametitle{Tecnologías a utilizar}
\begin{itemize}
\item JSON: Cómo almacenar los datos de sesión
\item Comenzando con PostgreSQL
\item Re-utilizando y escribiendo el código
\item Pasando a Redis\ldots
\end{itemize} 
\end{frame}

\subsection{Resultado}
\begin{frame}\frametitle{Resultado}
Un screenshot:
En github: https://github.com/javierwilson/flisol2012
\end{frame}

\end{document}
